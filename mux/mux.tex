\documentclass[a4paper,12pt]{article}
\usepackage[margin=2.5cm]{geometry}
\usepackage{tikz}
\usepackage{circuitikz}
\usepackage{graphicx}
\usepackage{amsmath}

\pagestyle{empty}

\begin{document}

\begin{center}
    {\LARGE \textbf{4-to-1 Multiplexer (4:1 MUX)}}
\end{center}

\vspace{1cm}

\begin{center}
    \section*{Circuit}
    \bigskip
    \begin{circuitikz}[american]
        \draw (0,3.3) node[left]{$I_0$} -- (2,3.3) node[circ] (I0) {};
        \draw (0,1.35) node[left]{$I_1$} -- (2,1.35) node[circ] (I1) {};
        \draw (0,-0.65) node[left]{$I_2$} -- (2,-0.65) node[circ] (I2) {};
        \draw (0,-2.7) node[left]{$I_3$} -- (2,-2.7) node[circ] (I3) {};
        
        \draw (0,-5) node[left]{$S_0$} -- (1.5,-5) node[circ] (S0) {};
        \draw (0,-7) node[left]{$S_1$} -- (2.5,-7) node[circ] (S1) {};

	\node[american and port, number inputs=3] (A0) at (8,3) {};
	\node[american and port, number inputs=3] (A1) at (8,1) {};
	\node[american and port, number inputs=3] (A2) at (8,-1) {};
	\node[american and port, number inputs=3] (A3) at (8,-3) {};


        \draw (I0) -- (A0.in 1);
        \draw (I1) -- (A1.in 1);
        \draw (I2) -- (A2.in 1);
        \draw (I3) -- (A3.in 1);

        \node[american not port] (N0) at (3.5,-5) {};
        \draw (S0) -- (N0.in);
        \draw (N0.out) -- ++(1,0) node[circ] (nS0) {};

        \node[american not port] (N1) at (3.5,-7) {};
        \draw (S1) -- (N1.in);
        \draw (N1.out) -- ++(0.5,0) node[circ] (nS1) {};

        \draw (nS1) |- (A0.in 2);
        \draw (nS0) |- (A0.in 3);

        \draw (nS1) |- (A1.in 2);
        \draw (S0) |- (A1.in 3);

        \draw (S1) |- (A2.in 2);
        \draw (nS0) |- (A2.in 3);

        \draw (S1) |- (A3.in 2);
        \draw (S0) |- (A3.in 3);

        \node[american or port] (O1) at (10,2) {};
        \draw (A0.out) |- (O1.in 1);
        \draw (A1.out) |- (O1.in 2);
        \node[american or port] (O2) at (10,-2) {};
        \draw (A2.out) |- (O2.in 1);
        \draw (A3.out) |- (O2.in 2);

        \node[american or port] (O3) at (11.5,0) {};
        \draw (O1.out) -- (O3.in 1);
        \draw (O2.out) -- (O3.in 2);

        \draw (O3.out) -- ++(1,0) node[right]{$Y$};
    \end{circuitikz}
    
\end{center}

\[
\boxed{Y = I_0\overline{S_1}\,\overline{S_0} + I_1\overline{S_1}S_0 + I_2S_1\overline{S_0} + I_3S_1S_0}
\]

\vspace{1cm}

\begin{center}
    \section*{Truth Table}
    \bigskip
    \scalebox{1.3}{
        \renewcommand{\arraystretch}{1.3} 
        \setlength{\tabcolsep}{12pt}
        \begin{tabular}{c c|c}
            \textbf{S$_1$} & \textbf{S$_0$} & \textbf{Y} \\ \hline
            0 & 0 & I$_0$ \\
            0 & 1 & I$_1$ \\
            1 & 0 & I$_2$ \\
            1 & 1 & I$_3$ \\
        \end{tabular}
    }
\end{center}

\end{document}
